% Appendix C

\chapter{Anleitung zur Verwendung} % Main appendix title

\label{UserManual} % For referencing this appendix elsewhere, use \ref{AppendixA}

\lhead{Appendix C. \emph{Anleitung zur Verwendung}} % This is for the header on each page - perhaps a shortened title

\textbf{Benötigte Pakete}
\begin{enumerate}
\item ANTLR 4.5 ...
\item SEWOL 1.0.0
\begin{enumerate}
\item TOVAL 1.0.0
\item JAGAL 1.0.0
\end{enumerate}
\end{enumerate}

\textbf{Es wurde auf folgendem System entwickelt und getestet}
\begin{enumerate}
\item Linux Ubuntu 14.04 LTS 64-Bit
\item Java version 1.7.0\_79
\item SWI-Prolog version 6.6.4 for amd64
\end{enumerate}

\textbf{Ausführen des Programms per Kommandozeile}\\
Entweder muss sich der Ordner libraries oder alle benötigten Pakete im Classpath befinden.\\
Alle Ordner (rule, log, output,..) müssen im bin-Ordner sein.\\

\texttt{java -cp .:../../libraries/* main.Main}



Das Paket \texttt{}

\textbf{Einrichten des Programms in Eclipse}\\


\textbf{Einbinden in eigenes Projekt}\\
Das Paket \texttt{iicmchecker} muss sich samt allen Abhängigkeiten im Build-Path des Projekts befinden. Gestartet wird es über das Kommando \texttt{run()}.\\
\begin{verbatim}
IICMChecker checker = new IICMChecker();
checker.run();
\end{verbatim}


\textbf{Individuelle Einstellungen}\\
Der IICMChecker besitzt diverse Methoden, um individuelle Einstellungen vorzunehmen.
Folgende Optionen sind möglich (im Kommentar steht die Klasse des Arguments):
\begin{verbatim}
checker.setLoggerLevel(loggerLevel);    	// java.util.logging.Level
checker.setRuleLocation(rulelocation);  	// java.lang.String
checker.setRuleFiles(rulefiles);  		// java.lang.String[]
checker.setLogLocation(loglocation);		// java.lang.String
checker.setLogFiles(logfiles);			// java.lang.String[]
checker.setOutputLocation(outputlocation);	// java.lang.String
\end{verbatim}

