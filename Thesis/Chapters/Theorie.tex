% Chapter Template

\chapter{Material und Methoden} % Main chapter title

\label{Chapter4} % Change X to a consecutive number; for referencing this chapter elsewhere, use \ref{ChapterX}

\lhead{Chapter 4. \emph{Material und Methoden}} % Change X to a consecutive number; this is for the header on each page - perhaps a shortened title

%-----------------------------------
%	THEORIE
%-----------------------------------
\section{Theorie}

\subsection{Ein praktisches Beispiel mit vielen Einschränkungen}
Bild zum Workflow + Erklärung\\
schriftliche Auflistung der Einschränkungen\\
-Intra-Instance \\
-Inter-Instance\\
--User U darf Task1 nicht mehr als 5 mal im Monat ausführen\\
--User U darf bei Task1 nicht mehr als 100000 Euro auszahlen\\
--U1 und U2 dürfen nicht mehr als 3mal an T1 und T2 zusammen arbeiten\\
-Inter-Process\\
--Ein User darf nicht mehr als 100 Tasks pro Tag erledigen\\
\subsection{Arten von Constraints - Herleitung}
zeitliche Beschränkungen
--absolute Einschränkungen\\
--relative Einschränkungen\\
Akkumulationen\\
Separation of Duty\\
Binding of Duty Constraints\\
Cardinality Constraints\\
Workflow Soundness\\
--Bild dazu\\


\subsection{Vorüberlegungen zur Grammatik}
Woran erkennt man den Kontext der Constraints? \\
Welches Rollenmodell wird benutzt, und welche Auswirkungen hat es auf die Grammatik?
\subsection{Formale Definition}
Zuerst Herleitung, was wir genau an Aussagen brauchen\\
Workflows\\
Zeitstempel\\
Constanten\\
Variablen\\
Literale\\
Regeln (Constraints sind Schlussfolgerungen, die sich aus Vorbedingungen ergeben.\\
Fakten\\
Status (aus Event Logs gelesen)\\
Ableitung,...\\
Zu Beispiel gefundene Einschränkungen definieren\\

Kann ein Task mehrere Instanzen haben?\\
Hierarchisches Rollenmodell\\
Umgang mit Negation\\
\subsection{Grammatik in BNF}
Konstanten sind als 'String' , Variablen ohne \\
Welche Zeichen darf man wo verwenden?\\
Verfügbare Literale\\
Syntax\\
Fakten: SET extern|workflow \\
Regeln: 	IF (..|..) (AND (..|..)*) \\
THEN (..|..) \\
WHERE t.name = t2.name AND  ...\\
\subsection{Erklärungen zur Grammatik}
Wenn sich etwas auf jede Instanz einzeln beziehen muss, muss Task1.workflow.instance = Task2.workflow.instance\\
Eignet sich für hierarchisches und auch für normales Rollenmodell. Hängt nur davon ab, ob die Hierarchie als Fakt gesetzt wurde.\\
User und ihre Rollenzuweisungen müssen nicht explizit angegeben werden. Nur wenn man das für die Vergleiche braucht.\\
Beziehung zwischen critical task pair und collaborateur.
\subsection{konkretes Beispiel}
Die zuvor gefundenen Beispiele wären mit der neuen Grammtik:\\
(C1): IF NUMBER OF USER executed TASK IS N AND N > 5 THEN USER cannot execute TASK
