% Chapter Template

\chapter{Material und Methoden} % Main chapter title

\label{Chapter4} % Change X to a consecutive number; for referencing this chapter elsewhere, use \ref{ChapterX}

\lhead{Chapter 4. \emph{Material und Methoden}} % Change X to a consecutive number; this is for the header on each page - perhaps a shortened title

%-----------------------------------
%	THEORIE
%-----------------------------------
\section{Theorie}

\subsection{Ein praktisches Beispiel mit vielen Einschränkungen}
Bild zum Workflow + Erklärung\\
schriftliche Auflistung der Einschränkungen\\
-Intra-Instance \\
-Inter-Instance\\
--User U darf Task1 nicht mehr als 5 mal im Monat ausführen\\
--User U darf bei Task1 nicht mehr als 100000 Euro auszahlen\\
--U1 und U2 dürfen nicht mehr als 3mal an T1 und T2 zusammen arbeiten\\
-Inter-Process\\
--Ein User darf nicht mehr als 100 Tasks pro Tag erledigen\\
\subsection{Arten von Constraints - Herleitung}
\subsubsection{sich gegenseitig ausschließende Tasks (Conflicting Tasks)}
TODO: kommt das zur Constraint Sammlung?
In manchen Fällen kann man Tasks nicht verschiedenen Rollen zuweisen  ohne die exisitierenden Rollen derart zu segmentieren, dass die schwer zu verwalten sind in Bezug auf das organsiatorische Modell. Außerdem können Rollenhierarchien dazu verwendet werden, in zwei verschiedenen Rollen zu agieren, die eigentlich getrennt waren. ZB könnte ein Manager als ein Clerk und gleichzeitig als sein eigenener Supervisor handeln.\\
Deswegen definieren wir \textbf{TC} $\subset$ T als eine Menge von \textbf{kollidierenden Tasks}. TC beinhaltet Tasks, deren Allokation von der Allokation von vorhergehend ausgeführten Tasks aus TC abhängt. Diese Abhängigkeit wird als Abhängigkeit zwischen Tasks aus TC beschrieben. $t_c\in TC$ gilt als entailed Task von $t_m\in TC$, wenn die Allokation von $t_n$ von der Allokation von $t_m$ eingeschränkt wird, mit $t_m<t_n$.
\cite{wolter_modeling_of_TBAC_in_BPMN}

\subsubsection{Entailment Constraint}
c=(TC, $n_u$,$m_{th}$) mit $n_u$ als minimale Anzahl an verschiedenen Usern, die einem Task zugewiesen werden müssen. Wenn $t_{ki}$ eine Instanz des Tasks $t_k$ ist, dann ist $m_th$ der Grenzwert von der Summe der Task Instanzen, denen ein User zugewiesen sein darf.
\cite{wolter_modeling_of_TBAC_in_BPMN}

zeitliche Beschränkungen
--absolute Einschränkungen\\
--relative Einschränkungen\\
Akkumulationen\\
Separation of Duty\\
Binding of Duty Constraints\\
Cardinality Constraints\\
Workflow Soundness\\
--Bild dazu\\


\subsection{Vorüberlegungen zur Grammatik}
Woran erkennt man den Kontext der Constraints? \\
Welches Rollenmodell wird benutzt, und welche Auswirkungen hat es auf die Grammatik?\\
Was passiert bei fehlenden Einträgen, oder falschen Einträgen?\\
Soll Negation erlaubt sein?\\
Was definiert der Timestamp, wenn es sich doch nur auf verschiedene Events bezieht?
\subsection{Basis von Werten, Aussagen...}
Zuerst Herleitung, was wir genau an Aussagen brauchen\\
\subsubsection{Parameter - Variablen und Konstanten}
\textbf{User}

\textbf{Rollen}

\textbf{zeitliche Parameter}

\subsubsection{Prädikate}
Prädikate sind Aussagen über bestimmte Zustände. Es gibt verschiedene Typen. \\
\begin{table}
\begin{tabular} {|c|c|c|}
Prädikat & Aurgumente & Beschreibung\\
\hline
\end{tabular}
\caption{sdf}
\label{tab:specification}
\end{table}

Statusprädikate sind Aussagen über Events

\subsubsection{Regeln}
Regeln (Constraints sind Schlussfolgerungen, die sich aus Vorbedingungen ergeben. Es gibt positive und negative. Die positiven sagen, dass etwas passieren muss, die negativen verbieten, dass etwas passiert)\\
Ableitung,...\\

\subsection{Grammatik in BNF}
Für ein besseres Verständnis der Definition erst einmal ein kurzes Beispiel:\\

\begin{verbatim}
SET 'Mark Maier' is related to 'Max Mueller'
IF USER_A executed 'Kredit beantragen' AND USER_A is related to USER_B
  THEN USER_B cannot execute 'Kredit prüfen'
\end{verbatim}
\begin{figure}[!h]
\caption{Beispiel Spezifikation einer einfachen Regel}
\label{fig:demorulefile}
\end{figure}

Konstanten sind als 'String' , Variablen ohne \\
Welche Zeichen darf man wo verwenden?\\
Verfügbare Literale\\
Syntax\\
Fakten: SET extern|workflow \\
Regeln: 	IF (..|..) (AND (..|..)*) \\
THEN (..|..) \\
WHERE t.name = t2.name AND  ...\\
\subsection{Erklärungen zur Grammatik}
Wenn sich etwas auf jede Instanz einzeln beziehen muss, muss Task1.workflow.instance = Task2.workflow.instance\\
Eignet sich für hierarchisches und auch für normales Rollenmodell. Hängt nur davon ab, ob die Hierarchie als Fakt gesetzt wurde.\\
User und ihre Rollenzuweisungen müssen nicht explizit angegeben werden. Nur wenn man das für die Vergleiche braucht.\\
Beziehung zwischen critical task pair und collaborateur.
\subsection{konkretes Beispiel}
Die zuvor gefundenen Beispiele wären mit der neuen Grammtik:\\
\begin{verbatim}
/* C1 */
IF NUMBER OF USER executed TASK IS N AND N > 5
  THEN USER cannot execute TASK
\end{verbatim}
\begin{figure}[!h]
\caption{Regeln für unsere gefundenen Beispiele}
\label{fig:resultrulefile}
\end{figure}
