% Chapter Template

\chapter{Verwandte Arbeit} % Main chapter title

\label{Chapter3} % Change X to a consecutive number; for referencing this chapter elsewhere, use \ref{ChapterX}

\lhead{Chapter 3. \emph{Introduction}} % Change X to a consecutive number; this is for the header on each page - perhaps a shortened title

%-----------------------------------
%	SECTION
%-----------------------------------

\section{Related work}


\subsection{Geschichte}
Chinese Wall Modell Bewer/ Nash\\
Business Rules und Auditierung ??
Um die Zuverlässigkeit von Abläufen zu gewährleisten, wird schon lange Auditierung betrieben. Ursprünglich ging es darum, den korrekten Ablauf nachzuweisen. Man kann damit aber auch die zulässige Ausführung von Ereignissen prüfen.
\subsection{Heute}
Übergang zu Inter/instance Constraints
Heute: verschiedene Forschungszweige: TBAC, Inter-Instance Constraints
Allgemein eher den Inhalt auswerten, statt eine Liste von verwandten Arbeiten anzugeben
Arten Unterscheiden, wie man Constraints spezifizieren kann
Als logische Prädikate, GL4
Welche Constraint-Richtungen werden behandelt?
ZB nur Entailment, Cardinality. 
\subsection{warner inter-Instance}
Diese Arbeit bezieht sich hauptsächlich auf die Arbeit von Warner und Atluri \cite{warner_inter_instance}.
In dieser Arbeit wird der theoretische Ansatz von Bertino et al. um inter instance constraints erweitert und ist der Ausgangspunkt meiner Arbeit. Allerdings gehen sie davon aus, dass die Constraints für Authorisierungszwecke gedacht sind, und unterscheiden deswegen zwischen statischer Analyse zur Entwurfszeit und dynamischer Analyse während des Ablaufs. Bei der dynamischen Analyse werden die Informationen nach jeder Aktion  aktualisiert und unter Umständen die Liste von verbotenen Nutzern / Rollen angepasst.
Da ich Auditierung benutze, wird statically checked weggelassen und weitere Regeln, die zur Laufzeit nötig sind.\\
Das Paper ist allerdings nur ein theoretischer Ansatz und behandelt zB die verschiedenen Eventtypen nicht ausreichen. Des weiteren wird nicht darauf eingegangen, was passiert, wenn  manche Informationen fehlen.
\subsection{leitner instance-spanning}
Hat ein Framework entwickelt.
\subsection{tan consistency}
Es geht eher um die Prüfung auf Konsistenz, es werden aber globale Constraints erwähnt.




\subsection{ProM - SCIFFchecker}
Sciff wurde von Marco Alberti, Federico Chesani und Marco Gavanelli im Rahmen des SOCS Projekts entwickelt. Es arbeitet ebenfalls mit Prolog, kann aber keine Inter Instance Constraints.



