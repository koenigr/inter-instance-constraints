% Chapter Template

\chapter{Praxis} % Main chapter title

\label{Chapter5} % Change X to a consecutive number; for referencing this chapter elsewhere, use \ref{ChapterX}

\lhead{Chapter 5. \emph{Ergebnisse}} % Change X to a consecutive number; this is for the header on each page - perhaps a shortened title

%----------------------------------------------------------------------------------------
%	PRAXIS
%----------------------------------------------------------------------------------------

\section{Implementierung}
\subsection{Algorithmus}
Zuerst aus Logs Status-Prädikate auslesen.\\
Für jedes critical task pair entsprechende collaborateurs setzen.\\
Schleife über Regeln\\
--Die resultierenden Heads bestimmen und schauen, ob es für cannot do ein executed und für must do kein executed gibt-> dann wurde es verletzt. Um das herauszufinden, wird die Regel in executed $ - >$ blabla, oder not(executed) $- >$ blabla übersetzt.\\
 
Einfaches Beispiel: Rule: executed(ui, t1)-> cannot do(ui,t2) und status: executed('tom', t1). Daraus wird abgeleitet: cannot do('tom', t2). Da es nicht in der DB ist, ist alles OK

\subsection{Aufbau mein Programm}

\begin{figure}[ht]
	\centering
  \includegraphics[width=0.9\textwidth]{"Figures/myProg"}
	\caption{Aufbau mein Programm}
	\label{fig: myprog}
\end{figure}

Das Programm liest mittels SEWOL MXML logs ein und übersetzt sie in Status Prädikate. 
Die Constraints werden ebenfalls eingelesen und in Regelprädikate übersetzt. Der Parser für die Grammatik wurde mit ANTLR erzeugt.
Die eigentliche Untersuchung kann entweder durch ein Skript oder innerhalb des Javacodes gestartet werden.

