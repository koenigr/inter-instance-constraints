% Chapter Template

\chapter{Entwicklung einer Definitionssprache für Regeln} % Main chapter title

\label{GrammatikKapitel} % Change X to a consecutive number; for referencing this chapter elsewhere, use \ref{ChapterX}

\lhead{Chapter 4. \emph{Grammatik}} % Change X to a consecutive number; this is for the header on each page - perhaps a shortened title

In diesem Kapitel wird die Grammatik vorgestellt, die es ermöglichen soll, Regeln innerhalb von Prozessinstanzen aber auch Instanzübergreifend zu definieren. Zuerst muss geklärt werden, welche Anforderungen an die Grammatik gestellt werden und welche Fragen auftauchen. Im zweiten Teil wird die Syntax und Semantik der Grammatik erläutert. Schließlich wird aufgezeigt, wie die vorgestellten Regeln aus Kapitel \ref{sec:exampleconstraints} mithilfe der zuvor definierten Grammatik beschrieben werden können.

%
% Anforderungen an die Grammatik
%
\section{Anforderungen an die Grammatik}
Um eine ausdrucksstarke Grammatik zu definieren, die möglichst viele Fälle abdeckt, muss zuerst untersucht werden, welche genauen Anforderungen an sie gestellt werden.

Grundsätzlich ist das Ziel, Regeln zur Ausführung bestimmter Aktivitäten auf Basis bereits abgeschlossener Aktivitäten aufzustellen. Es muss also einen Teil geben, in dem man die Bedingungen festlegen kann um dann anzugeben, welche gewünschte Aktion daraus resultiert. Die Bedingungen bilden in den meisten Fällen eine Konjunktion. Werden alle Bedingungen erfüllt, tritt die Regel in Kraft. Zur Vollständigkeit wird hier auch die Option mit aufgenommen, absolute Regeln ohne Bedingungen aufzustellen, die in jedem Fall gelten sollen.

Für alle Regeltypen, die in \ref{sec:ArtenConstraints} gefunden wurden, ist Konjunktion von positiven Bedingungen ausreichend. Um dem Nutzer größtmögliche Freiheit zu lassen, sollte Konjunktion und Negation von Bedingungen ebenfalls angeboten werden.

Das Ziel der Arbeit ist es, Regeln in Instanz-übergreifendem Kontext aufstellen zu können. Trotzdem muss es auch möglich sein, Regeln wie in bisherigen Compliancecheckern auch innerhalb einer Instanz zu prüfen. Es ist deswegen eine eindeutige Konvention notwendig, die deutlich macht, in welchem Kontext die Regel arbeitet.

Im Rollenbasierten Authorisierungsmodell existiert das einfache Rollenmodell, in welchem dem Nutzer nur die Rollen zur Verfügung stehen, die ihm explizit zugewiesen wurden. Es gibt keine eindeutig festgelegten Hierarchien zwischen den Rollen. Im hierarchischen Rollenmodell hingegen kann ein Nutzer jede Rolle annehmen, die gleich oder unter der ihm zugewiesenen Rolle steht. Die Grammatik sollte beide Modelle behandeln können.

Da es Bedingungen in Bezug auf Zeitpunkte, Zeitunterschiede und Werte von Attributen gibt, müssen zumindest grundlegende arithmetische Operationen erlaubt sein.

Eventlogs speichern zu jeder Aktivität jedes einzelne Event. Das könnte entweder die Eingabe der Regeln unnötig kompliziert machen (indem der Anwender zu jeder einzelnen Aktivität immer das Event angeben muss) oder es würde zu multiplen Ergebnissen führen. Es muss eine Einigung geben, ob Events explizit angegeben werden müssen, oder sich Prädikate wie \texttt{timestamp(TASK)} auf ein konkretes Event bezieht.

Unter Berücksichtigung aller gestellten Forderungen sollte trotzdem eine leicht zu lernende und leicht zu verstehende Notation gewährleistet sein. Im folgenden wird die entwickelte Grammatik vorgestellt und im Anschluss geklärt, inwiefern die hier aufgeführten Fragen und Anforderungen gelöst wurden.
%
% Definition der Grammatik
%
\section{Definition der Grammatik}
Für ein besseres Verständnis der Definition wird hier zuerst ein kleines, intuitiv zu verstehendes Beispiel aufgezeigt (Abb. \ref{fig:demorulefile}). Am Anfang wird spezifiziert, dass Mark Maier und Max Mueller Verwandte sind. In der nächsten Zeile wird die Regel aufgestellt, dass Verwandte nicht gemeinsam and den Aktivitäten 'Kredit beantragen' und 'Kredit prüfen' arbeiten dürfen. Sollte Mark Maier den Kredit beantragt haben, wird Max Mueller die Prüfung des Kredits untersagt.\\

\begin{verbatim}
SET 'Mark Maier' is related to 'Max Mueller'

DESC "'Kredit beantragen' und 'Kredit prüfen' dürfen 
	nicht von Verwandten ausgeführt werden"
IF USER_A executed 'Kredit beantragen' AND USER_A is related to USER_B
  THEN USER_B cannot execute 'Kredit prüfen'
\end{verbatim}
\begin{figure}[!h]
\caption{Beispiel Spezifikation einer einfachen Regel}
\label{fig:demorulefile}
\end{figure}


Das Ziel der Grammatik ist es, gewünschte oder unerwünscht Aktionen in Abhängigkeit von zuvor stattgefundenen Aktivitäten zu definieren. Zu diesem Zweck werden die Bedingungen als eine Konjunktion von Prädikaten über den Verlauf gebildet. Sollten diese Aussagen alle zu einem positiven Ergebnis führen, tritt die Einschränkung in Kraft. Diese Einschränkung macht eine Aussage darüber, ob eine bestimmte Aktivität von einem Nutzer / Rolle ausgeführt werden muss bzw dass sie von einem Nutzer/Rolle nicht ausgeführt werden darf. Es gibt auch Regeln, die aufzeigen, wann ein Verlauf nicht der Spezifikation entsprach (\texttt{illegal\_execution}).\\
In den nächsten Abschnitten werden zuerst die Variablen, Konstanten und Prädikate vorgestellt, bevor dann im Anschluss die Syntax und Semantik der Grammatik genauer erläutert wird.

\subsection{Argumente - Variablen und Konstanten}

Prädikate sind Aussagen über Parameter einer Aktivität oder über Beziehungen zwischen Rollen oder Nutzern. Außer dem Prädikat \texttt{illegal execution} muss jedem mindestens ein Argument übergeben werden. Die Argumente sind entweder Variablen oder Konstanten in Form einer Zeichenkette oder einem numerischen Wert. Der Typ des Argumentes wird in der Grammatik nicht explizit deklariert, sondern erschließt sich aus dem Kontext des jeweiligen Prädikates. In diesem Abschnitt werden alle verwendbaren Typen vorgestellt. Sollte ein Argument als Variable übergeben werden, beginnt das Argument für jeden Typ mit einem Großbuchstaben. Die Form der Konstanten hängt von dem entsprechenden Typ ab. Grundsätzlich gilt, dass alle Typen außer den Zeiten und numerischen Werten einen String als Konstante haben, der mit einfachen Anführungszeichen umschlossen sein muss. Innerhalb der Anführungszeichen sind alle Zeichen erlaubt.

\begin{table}[h]
\begin{tabular} {|p{2cm}|p{13cm}|}
\hline
\textbf{Typ}&\textbf{Beschreibung}\\
\hline
UT& Variablen und Konstanten über Nutzer. Als Konstante sind Nutzer ein String.\\
\hline
RT& Variablen und Konstanten über Rollen. Als Konstante sind Rollen ein String.\\
\hline
TT& Variablen und Konstanten über Aktivitäten. Als Konstante sind Aktivitäten ein String.\\
\hline
WT& Variablen und Konstanten über Prozesse. Dieser Typ bezeichnet das Prozessschema. Als Konstante sind Prozesse ein String.\\
\hline
WIT& Variablen und Konstanten über Prozessinstanzen. Als Konstante sind Prozessinstanzen ein String.\\
\hline
ET&  Variablen und Konstanten über Eventtypen. Als Konstante sind Events Elemente aus der Menge $\{$'started', 'completed',..$\}$ (siehe \ref{sec:activities}).\\
\hline
TP& Variablen und Konstanten über Zeitpunkte. Als Konstante Zeitpunkt nach ISO 8601.\\
\hline
TS& Variablen und Konstanten über Zeitspannen. Als Konstante Zeitspanne nach ISO 8601.\\
\hline
NT&  Variablen und Konstanten über numerische Werte. Als Konstante sind numerische Werte eine Zahl größer Null. Negative Werte sind nicht erlaubt.\\
\hline
\end{tabular}
\caption{Argument Typen, die bei Prädikaten vorkommen können}
\label{tab:args}
\end{table}

Zeitpunkte nach ISO 8601 werden im Format JJJJ-MM-DD'T'hh:mm:ss.f , wobei f ein dezimaler Bruchteil für Sekunden beliebiger Genauigkeit ist. Die Datums und Uhrzeit-Angabe wird von einem 'T' getrennt. Die Werte werden ohne Leerzeichen notiert. Es ist erlaubt, Werte mit geringerer Genauigkeit anzugeben, jedoch darf die Angabe immer nur von rechts (mit dem kleinsten Wert beginnend) weggelassen werden. Gültige Beispiele sind \texttt{2015-08-21T11:23:45.23526} , \texttt{2015-08-21T11}, \texttt{2015-08}. Eine ungültige Angabe hingegen wäre \texttt{2015-21T23}. Hier wurde der Monat und die Uhrzeit weggelassen, obwohl dahinter noch das Datum und die Minuten standen.

Zeitspannen im ISO 8601 Format sind 'P'JJJJ'Y'MM'M'DD'D''T'hh'h'mm'm'ss.f's'. Die Zeitspanne beginnt mit einem vorangestellten 'P', dahinter folgen Werte für Jahr, Monat, Tag, Stunden, Minuten, Sekunden. Datum und Zeitangaben werden hier ebenfalls voneinander mit dem Trennsymbol 'T' abgegrenzt. Im ISO 8601 sind auch Wochenangaben erlaubt. Da diese sich aber auch durch Monate und Tage darstellen lassen, wird hier darauf verzichtet. Bei Zeitspannen dürfen auch Werte in der "Mitte" weggelassen werden. Um trotzdem eindeutig identifizieren zu können, welche Einheit eine Angabe besitzt, muss hinter jedem Wert das zugehörige Symbol stehen: \textbf{Y} für Jahr, \textbf{M} Monat, \textbf{D} Tage, \textbf{h} Stungen, \textbf{m} Minuten, \textbf{s} Sekunden. Beispiele für Zeitspannen sind \texttt{P2Y6M1DT16h35m2s} (2 Jahre 6 Monate 1 Tag 16 Stunden 35 Minuten 2 Sekunden), \texttt{P1D} (1 Tag) und \texttt{P2Y1DT16h35m2s} (2 Jahre 1 Tag 16 Stunden 35 Minuten 2 Sekunden).

Zeitzonenangaben sind nicht erlaubt.

\subsection{Prädikate}
In diesem Abschnitt werden alle Verfügbaren Prädikate vorgestellt. Es gibt  7 verschiedene Typen: \textit{Externe Informationen, Spezifikation des Prozesses und der Authorisierung, Status, Prädikate für den Kopf einer Regel, Aggregationsprädikate, Vergleiche und arithmetische Operationen}. \\

Externe Informationen (Tabelle \ref{tab:extern}) sind Aussagen, die nicht direkt mit der Workflow Spezifikation zu tun haben aber dennoch relevant für den Ablauf sein könnten. Diese Prädikate müssen explizit in den Regeln gesetzt werden, da sie nicht aus den Logs herausgelesen werden und bei einer Anfrage immer \textit{false} zurückgeben würden.
\begin{table}[h]
\begin{tabular} {|p{6cm}|p{10cm}|}
\hline
\textbf{Prädikat} & \textbf{Beschreibung}\\
\hline
UT 'is related to' UT 		& Beide User sind verwandt \\
\hline
UT 'is partner to' UT		& Beide Akteure sind Partner \\
\hline
UT 'is in same group as' UT	& Beide Akteure sind in der selben Gruppe, Abteilung\\
\hline
\end{tabular}
\\\\\small Die Beschreibungen sind nur Vorschläge für den Einsatz. Dem Programmierer ist selbst überlassen, wie er diese Prädikate interpretieren und einsetzen möchte.
\caption{Prädikate für externe Informationen.}
\label{tab:extern}
\end{table}

Spezifikationsprädikate (Tabelle \ref{tab:specification}) bestimmen die Beziehungen und Zugehörigkeit zwischen Nutzern, Rollen und Tasks. Um korrekte Ergebnisse zu erhalten, sollten sie genauso gesetzt werden wie in der Spezifikation der Authorisierung zur Ausführungszeit des Prozesses.
\begin{table}[h]
\begin{tabular} {|p{6cm}|p{10cm}|}
\hline
\textbf{Prädikat} & \textbf{Beschreibung}\\
\hline
'role' RT 'can execute' TT	& RT ist in  R(TT)\\
\hline
'user' UT 'can execute' TT 	& UT ist in U(TT)\\
\hline
'user' UT 'belongs to role' RT  & (UT,RT) ist in UR\\
\hline
RT 'is glb of' TT 		& greatest lower bound. TT muss mindestens mit Rolle RT ausgeführt werden\tnote{1}\\
\hline
RT 'is lub' TT 			& lowest upper bound. TT darf höchstens mit Rolle RT ausgeführt werden\tnote{1}\\
\hline
RT 'dominates' RT 		& Rolle 1 (links) dominiert Rolle 2 (rechts)\\
\hline
'critical{\_}task{\_}pair(' TT ',' TT ')'& Die beiden Tasks sind ein kritisches Paar. Die User werden markiert, die dieses Paar ausführen\\
\hline
\end{tabular}
\\\\\small Prädikate für die Spezifikation der Authorisierung, Rollenbeziehungen und Bestimmung kritischer Aufgabenpaare.
\caption{Spezifikation}
\label{tab:specification}
\end{table}

Statusprädikate (Tabelle \ref{tab:status}) sind Aussagen über Aktivitäten. Diese können in den Bedingungen einer Regel eingesetzt werden und werden später mit den Informationen aus den Logs verglichen. 
\begin{table}[h]
\begin{tabular} {|p{6cm}|p{10cm}|}
\hline
\textbf{Prädikat} & \textbf{Beschreibung}\\
\hline
('user')? UT 'executed' TT      & Ut hat TT ausgeführt. Der Event-Typ ist unbestimmt und muss extra angegeben werden.\\
\hline
'role' RT 'executed' TT		& RT hat TT ausgeführt. Der Event-Typ ist unbestimmt und muss extra angegeben werden.\\
\hline
UT 'is assigned to' TT		& UT wurde TT zugewiesen. Entspricht dem Event-Typ 'assign'.\\
\hline
TT 'started'			& TT wurde gestartet. Entspricht dem Event-Typ 'start'.\\
\hline
TT 'aborted'			& TT wurde abgebrochen. Entspricht dem Event-Typ 'ate\_abort'.\\
\hline
TT 'completed'			& TT wurde erfolgreich abgeschlossen. Entspricht dem Event-Typ 'complete'. \\
\hline
'EventType(' TT ')'.$<$event$>$ 	& Prädikat, um einer Aktivität einen beliebigen Event-Typ zuweisen zu können. $<$event$>$ ist dabei eine Konstante aus den 12 zuvor bestimmten Events.\\
\hline
UT 'is collaborator of' UT	& UT sind alle Akteure, die an criticalTaskPair gearbeitet haben. Collaborator muss nicht extra wie criticalTaskPair gesetzt werden, sondern wird in der Analysephase vom ModelChecker berechnet.\\
\hline
'timestamp of' TT 'is' VAR	& Der Zeitpunkt von TT wird in eine Variable an der Stelle von VAR geschrieben.\\
\hline
'timeinterval of' TT 'and' TT 'is' VAR 	& Das Zeitinterval zwischen TT 1 (links) und TT 2 (rechts) wird in eine Variable an der Stelle VAR geschrieben.\\
\hline
'attribute' VAR1 'of' TT 'is' VAR2 & TODO\\
\hline	
\end{tabular}
\\\\\small Da sich die beiden \textit{executed} Prädikate nicht auf ein bestimmtes Event beziehen, ist es notwendig, das Event extra anzugeben, um keine multiplen Ergebnisse zu erhalten. Es werden vier Prädikate für die Event-Typen \textit{assign}, \textit{start}, \textit{abort} und \textit{complete} angeboten. Für alle weiteren kann man das allgemeine Event-Prädikat 'Event(' TT ')' verwenden. Für das Event \textit{skip} wäre es EventType(TT).'skip'. Das Event muss in zwei einfachen Anführungszeichen stehen.
\caption{Prädikate, um Aussagen über den Status in die Regeln mit einbeziehen zu können}
\label{tab:status}
\end{table}

Aggregations Prädikate (Tabelle \ref{tab:conditional}) (\textit{NUMBER, SUM, AVG, MIN, MAX}) geben einen Wert über die Aggregation von einer Variablen zurück, die in den Klauseln des Körpers dieses Prädikates stehen. \texttt{MIN OF X WHERE 'T1' completed AND timestamp of 'T1' is X IS N} berechnet das Minimum von allen Zeitpunkten, an denen die Aktivität 'T1' abgeschlossen wurde und schreibt es in die Variable N. X ist die Variable, mit der gerechnet wird und mindestens ein Mal in Körper dieses Prädikates vorkommen muss. Die beiden Literale \textbf{'T1' completed} und \textbf{timestamp of 'T1' is X} bilden den Körper. '\textbf{MIN OF} Var \textbf{WHERE} body \textbf{IS} Var ist das Aggregations-Prädikat selbst.
\begin{table}[h]
\begin{tabular} {|p{6cm}|p{10cm}|}
\hline
\textbf{Prädikat} & \textbf{Beschreibung}\\
\hline
'NUMBER WHERE' $<$body$>$ 'IS' RES		& Zählt die Anzahl der verschiedenen Lösungen für $<$body$>$ und speichert sie in einer Variablen an der Stelle von RES \\
\hline
'NUMBER OF' VAR 'WHERE' $<$body$>$ 'IS' RES	& Zählt die Anzahl der verschiedenen Lösungen für VAR, die in $<$body$>$ vorkommen, wenn $<$body$>$ erfüllt wird. VAR kann eine beliebige Variable sein, muss aber mindestens einmal in $<$body$>$ vorkommen.\\
\hline
'SUM OF' NT$|$TP$|$TS 'WHERE' $<$body$>$ 'IS' RES		& Gibt die Summe einer Variablen aus $<$body$>$ zurück. Diese Variable sollte für einen numerischen Wert stehen (zB in den Attributen einer Aktivität) oder für Zeitpunkte oder Zeitintervalle.\\
\hline
'AVG OF' NT$|$TP$|$TS 'WHERE' $<$body$>$ 'IS' RES		& Gibt den Durchschnitt einer Variablen aus $<$body$>$ zurück.\\
\hline
'MIN OF' NT$|$TP$|$TS 'WHERE' $<$body$>$ 'IS' RES		& Gibt das Minimum einer Variablen aus $<$body$>$ zurück.\\
\hline
'MAX OF' NT$|$TP$|$TS 'WHERE' $<$body$>$ 'IS' RES		& Gibt das Maximum einer Vairablen aus $<$body$>$ zurück.\\
\hline
\end{tabular}
\\\\\small Das Resultat wird in der Variable gespeichtert, die anstelle von RES definiert wurde. $<$body$>$ ist eine Konjunktion von Status- , Externen und Spezifikationsprädikaten. Für $<$body$>$ gelten die selben Regeln wie für den Körper einer Regel bezüglich verwendbarer Prädikate, Negation und Disjunktion. 
\caption{Aggregationsprädikate}
\label{tab:conditional}
\end{table}


Vergleiche (Tabelle \ref{tab:comparison}) dienen dazu, festzustellen, in welchem Verhältnis zwei Werte zueinander stehen. Es können immer nur zwei Werte des gleichen Typs verglichen werden. Bei Gleichheitsabfragen sind alle Typen erlaubt, bei Abfragen über Größenverhätltnisse sind außer den numerischen Werten und Zeiten auch Rollen bezüglich der Hierarchie erlaubt.
\begin{table}[h]
\begin{tabular} {|p{6cm}|p{10cm}|}
\hline
\textbf{Prädikat} & \textbf{Beschreibung}\\
\hline
 $= | !=$		& Gleichheit und Ungleichheit kann man auf alle Argumenttypen anwenden, solange auf der rechten und linken Seite der selbe Typ steht. Es können jeweils Strings, numerische Werte, Zeitpunkte und Zeitstempel untereinander verglichen werden. \\
\hline
 $< | <= | > | >=$   	& Ungleichheit kann nur auf Rollen (um Beziehungen zwischen Rollen bezüglich der Hierarchie festzustellen), numerische Werte, Zeitpunkte und Zeitstempel angewendet werden. \\
\hline
\end{tabular}
\\\\\small 'Manager' $>$ 'Azubi' ergibt ein positives Ergebnis, wenn der 'Manager' in der Hierarchie über dem 'Azubi' steht.
\caption{Vergleiche}
\label{tab:comparison}
\end{table}

Arithmetische Operationen (Tabelle \ref{tab:operations}) sind grundsätzlich nur für numerische Werte erlaubt. Eine Ausnahme bildet die Summe aus einem Zeitpunkt und einer Zeitspanne (Das Ergebnis ist ein Zeitpunkt) und die Differenz aus zwei Zeitpunkten (das Ergebnis ist eine Zeitspanne).
\begin{table}[h]
\begin{tabular} {|p{6cm}|p{10cm}|}
\hline
\textbf{Prädikat} & \textbf{Beschreibung}\\
\hline
 $+$ 		& Numerische Werte und TP + TS (= TP) \\
\hline
 $-$		& Numerische Werte und TP - TP (= TS) \\
\hline
 $ * | / $   	& Als Argumente sind nur numerische Werte erlaubt. \\
\hline
\end{tabular}
\\\\\small 2015-08-03 + P5D liefert das Ergebnis 2015-08-08.
\caption{Arithmetische Operationen}
\label{tab:operations}
\end{table}
\newpage
Kopfprädikate bestimmen das gewünschte Resultat einer Regel. Sie machen Aussagen darüber, ob eine Aktivität von einem Nutzer/Rolle ausgeführt werden darf oder muss. Diese müssen im Kopf einer Regel stehen, und dürfen nicht im Körper vorkommen. \texttt{illegal execution} kann dazu verwendet werden, eine Ausführung als ??? (TODO) zu kennzeichnen, sofern die Spezifikation im Körper der Regel erfällt wurde. (Tabelle \ref{tab:head})
\begin{table}[h]
\begin{tabular} {|p{6cm}|p{10cm}|}
\hline
\textbf{Prädikat} & \textbf{Beschreibung}\\
\hline
UT cannot execute TT		& Nutzer UT darf TT nicht ausführen. \\
\hline
UT must execute TT  		& Nutzer UT muss TT ausführen. \\
\hline
RT cannot execute TT		& Rolle RT darf TT nicht ausführen.\\
\hline
RT must execute TT		& Rolle RT muss TT ausführen.\\
\hline
illegal execution		& Die Prozessinstanz hat den Körper der Regel erfüllt (was aber nicht erwünscht war) und hat somit die Spezifikation gebrochen.\\
\hline
\end{tabular}
\\\\\small Im Körper von \texttt{illegal execution} kann alles beschrieben werden, was zu einer ?? Ausführung führt. Wenn \texttt{IF USER executed 'T1' AND USER executed 'T2' THEN illegal execution} \textit{true} zurückgibt, bedeutet es, dass ein Nutzer sowohl T1 als auch T2 ausgeführt hat, dies aber nicht erlaubt war.
\caption{Prädikate für den Kopf einer Regel}
\label{tab:head}
\end{table}



\subsection{Regeln}
Regeln (Constraints sind Schlussfolgerungen, die sich aus Vorbedingungen ergeben. Es gibt positive und negative. Die positiven sagen, dass etwas passieren muss, die negativen verbieten, dass etwas passiert)\\
Ableitung,...\\
Die Köper der Regel werden als Konjunktion (Disjunktion ist ebenfalls möglich, jedoch nicht zwingend notwendig - siehe Kapitel \ref{sec:disjunction}) von Prädikaten gebildet.\\

\begin{verbatim}
IF body THEN head
\end{verbatim}

Der Körper - body ...\\
Der Kopf der Regel - head darf nur aus 

\subsection{Grammatik - Syntax und Semantik}

\textbf{Reservierte Keywords}\\




Konstanten sind als 'String' , Variablen ohne \\
Welche Zeichen darf man wo verwenden?\\
Verfügbare Literale\\
Syntax\\
Fakten: SET extern|workflow \\
Regeln: 	IF (..|..) (AND (..|..)*) \\
THEN (..|..) \\
WHERE t.name = t2.name AND  ...\\


%
% Erklärungen zur Grammatik
%
\section{Diskussion zur Grammatik}
Wenn sich etwas auf jede Instanz einzeln beziehen muss, muss Task1.workflow.instance = Task2.workflow.instance\\
Eignet sich für hierarchisches und auch für normales Rollenmodell. Hängt nur davon ab, ob die Hierarchie als Fakt gesetzt wurde.\\
User und ihre Rollenzuweisungen müssen nicht explizit angegeben werden. Nur wenn man das für die Vergleiche braucht.\\
Beziehung zwischen critical task pair und collaborateur.
\subsection{Disjunktion}
\label{sec:disjunction}
Es wird wenige Fälle geben, in denen eine Disjunktion von Klauseln notwendig ist. Um besser nachvollziehen zu können, zu welchem Fall eine Regelverletzung gehört, ist es oft sogar sinnvoller, eine Regel, die mehrere Fälle erlaubt, und mehrere Regeln aufzuteilen. Jedoch kann es verwendet werden, um Kardinalitätsaussagen einfacher darstelen zu können (TODO: mein Beispiel auf einem Zettel suchen), deswegen wird Disjunktion in der Grammatik erlaubt. Um die ??Assoziativität?? deutlich zumachen, müssen Disjunktionen in einer Klammer stehen. Disjunktion ohne umgebende Klammern ist nicht erlaubt. Disjunktionen von Konjunktionstermen ist nicht erlaubt.

\begin{verbatim}
IF (User executed 'task1' OR USER executed 'task2')
THEN User cannot execute 'task3'
\end{verbatim}
\begin{figure}[!h]
\caption{Disjunktion im Körper einer Regel. Man beachte, dass jede Disjunktion von umschließenden Klammern umgeben sein muss.}
\label{fig:disjunction1}
\end{figure}

Im Kopf einer Regel ist Disjunktion nicht erlaubt. Die Regel muss in zwei getrennte Regeln gespalten werden.

\begin{verbatim}
IF User executed 'task1'
THEN User cannot execute 'task2'

IF User executed 'task1'
THEN User cannot execute 'task3'
\end{verbatim}
\begin{figure}[!h]
\caption{Disjunktion im Kopfbereich ist nicht erlaubt. Es müssen zwei getrennte Regeln erstellt werden.}
\label{fig:disjunction2}
\end{figure}

\subsection{Umgang mit verschiedenen Rollenmodellen}
In den meisten Systemen wird ein hierarchisches Rollenmodell eingesetzt. Man ist bei der Analyse mit diesem Modelchecker nicht daran gebunden.


\subsubsection{Definition eigener Prädikate}
Es kann nötig erscheinen, eigene Prädikate zu definieren. Zum Beispiel um weitere Personengruppen anzulegen oder um Abkürzungen für Zusammenhänge zu erstellen. Dafür gibt es die Möglichkeit, ein Prädikat mittels des Schlüsselwortes "Def" zu Beginn der Regeldefinitionen zu setzen. In der Definition wird der Typ der Argumente gesetzt. Man hat die Wahl zwischen UT,RT,TT,...
Um die eigenen Prädikate nutzen zu können, muss man jedoch beachten, dass sie bei der Definition noch keinen "Wert" haben. Entweder muss man sie dann mit "SET" setzen oder mann kann ihnen eine Regel erstellen. Um die Prädikate für den Parser und die weitere Verarbeitung eindeutig sind, sind sie an eine spezielle Form gebunden: predicate{\_}name( ARGType,...).

\begin{verbatim}
DEF suspicious(UT)
DEF skipped(TT)
DEF aborted(UT,TT)

SET suspicous('Max Neuer')
SET suspicous('Tom Weisser')

IF EventType(ACTIVITY).'skipped' THEN skipped(ACTIVITY)
IF ACTOR executed ACTIVITY AND EventType(ACTIVITY).'aborted' THEN aborted(ACTOR, ACTIVITY)
\end{verbatim}
\begin{figure}[!h]
\caption{Definition eigener Prädikate. .. wurde gesetzt, .. hat eine eigene Regel}
\label{fig:define}
\end{figure}

\subsection{Negation}
Negation wird ebenfalls für die Bildung der meisten Regeln nicht benötigt, wird hier aber der Vollständigkeit halber erlaubt. Dabei gilt hier das Prinzip der \textbf{negation as failure}. Da nicht garantiert werden kann, dass ein Arbeitsablauf vollständig und korrekt geloggt wurde, muss man sich hier darauf einigen, dass das "nicht-vorhandensein" einer Klausel trotzdem mit der Negation dieser Klausel gleichzusetzen ist. Sollte das ein Fehler sein, entsteht ein False-positive Eintrag, dh. es wird ein Fehler zuviel angezeigt.

\textbf{Wie werden die Schutzziele erfüllt?}

\textbf{Kann man alle Constraints definieren?}

%
% Nochmal konkretes Beispiel
%
\section{konkretes Beispiel}
Die zuvor gefundenen Beispiele wären mit der neuen Grammtik:\\
\begin{verbatim}
/* C1 */
IF NUMBER OF USER executed TASK IS N AND N > 5
  THEN USER cannot execute TASK
\end{verbatim}
\begin{figure}[!h]
\caption{Regeln für unsere gefundenen Beispiele}
\label{fig:resultrulefile}
\end{figure}
